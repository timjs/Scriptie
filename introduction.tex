\startcomponent introduction
\product Thesis
\environment ThesisLayout

\chapter{Introduction}

The starting point of this thesis are the articles of Holger Bech Nielsen, professor at the Niels Bohr Institute at the University of Copenhagen. Nielsen is know by his publications about particle physics and string theory. In 1969 he proposed, independently of Yoichiro Nambu and Leonard Susskind, that the Veneziano model is a theory of strings. He is also known by several concepts in quantum mechanics, named after him.

In the past five years, Nielsen also published a series of articles on imaginary parts in the Lagrangian \cite[Nielsen:2006un] and complex action \cite{Nielsen:2007ut}. He uses this theory to predict the future of the \LHC\ \cite{Nielsen:2008uy,Nielsen:2008uk} and Higgs broadening \cite{Nielsen:2007ut}.

In this thesis we study the foundation of Nielsen's complex action theory. Can there be something like a \emph{complex action}? What are the implications for the \emph{imaginary part of the Lagrangian}? And the most important: does the theory hold in the \emph{classical limit}?

We will test the complex action theory in two different cases.
The first one is classical. We start with the principle of a Lagrangian with an imaginary part and derive the quantum mechanic equation of motion: the time dependant Schrödinger equation.
The second one is a derivation from quantum to classical. Starting with the Feynman path integral, with complex action, we try to derive the classical equations of motion.
In the mean time we discuss the physical implications of a complex Lagrangian and action.

Though a couble of Nielsen's publications are very controversial.
\todo{Which ones?}
\todo{Why?}
In this thesis we discuss some of the underlying concepts of his theory, in particular the notion of a complex Lagrangian.

\section{Things that should be added}

\startitemize
\item \todo{More explanation on physics.}
\item \unknown
\stopitemize

\cite[url1], \cite[url2], \cite[url3], \cite[url4], 

\placepublications[criterium=all]

\stopcomponent

% vim: spell spelllang=en

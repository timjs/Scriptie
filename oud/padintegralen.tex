\startcomponent padintegralen
\product        scriptie
\environment    thesislayout

\chapter[chp:paden]{Oneindig veel paden}

\startemphasize
In dit hoofdstuk introduceren we de Feynman padintegraal, een alternatieve methode om kwantummechanische berekeningen uit te voeren. Allereerst proberen we met een heuristische afleiding een gevoel te geven voor de herkomst van de padintegraal, waarna we een voorbeeld bekijken waarin we de padintegraal zullen gebruiken.

Wanneer we een goed beeld hebben kunnen we over gaan tot de concrete afleiding van Feynman's padintegraal. Deze geven we met behulp van \unknown. Hierna bekijken we enkele oplos methodes. In het bijzonder leiden we af wanneer we een padintegraal exact kunnen oplossen door deze te schrijven in een \emph{additieve vorm}. Daarna bestuderen we benaderingsmethoden zoals Fourierdecompositie.
\stopemphasize

\startemphasize
In dit eerste hoofdstuk geven we een inleiding tot de \emph{veel paden methode} in de kwantummechanica. Deze alternatieve kijk op de kwantummechanica levert niet alleen een andere manier van rekenen op. Het levert ons ook een andere blik op 
\stopemphasize

\placeintermezzo
  {Richard P. Feynman}
\startintermezzo
\subject{Richard P. Feynman\\ (1918--1988)}

Een Amerikaans natuurkundige, het grootste deel van zijn leven werkzaam bij het \infull{CALTECH} (\CALTECH). In 1965 ontving hij de Nobel prijs (samen met Sin-Itiro Tomonaga en Julian Schwinger) voor zijn werk in kwantum elektro dynamica (\QED). Daarnaast is hij bekend van de Feynmandiagrammen binnen de subatomaire fysica.
\stopintermezzo

\subsection{Heuristische afleiding Schrödingervergelijking}

Bekijk een algemene vlakke golf
  \startformula
  \Psi = A \e^{i(\vec{k} \cdot \vec{r} - \omega t)}.
  \stopformula
Met de Einstein en De Broglie relaties
  \startformula\startspread
  E = \hslash \omega  \SR
  \vec{p} = \hslash \vec{k}
  \stopspread\stopformula
Kunnen we \see[for:Vlakke golf] omschrijven in
  \startformula
  \Psi = A \e^{i(\vec{p} \cdot \vec{r} - E t) / \hslash}.
  \stopformula
Afleiden naar ruimte en tijd geeft
  \startformula
  .
  \stopformula
Zodat
  \startformula
  .
  \stopformula
Dit leid direct tot de Schrödingervergelijking
  \startformula
  i \hslash \pdiff{\Psi}{t} = - \frac{\hslash}{2m} \nabla^2 \Psi + V \Psi.
  \stopformula

Het idee van Feynman is om een deeltje dat vertrekt in punt $A$ en aankomt in punt $B$ niet één pad aflegt, maar alle mogelijke paden tussen $A$ en $B$ kan afleggen. Deze paden hoeven niet aan fysische wetten te voldoen zoals de Tweede Wet van Newton of het Principe van Hamilton en kunnen met gelijke kans worden gekozen.

\subsection[sec:heurpad]{Heuristische afleiding padintegraal}

In \unknown\ kwam Richard Feynman met een alternatieve methode om kwantummechanische berekeningen uit te voeren. In plaats van \unknown\ 
Om een gevoel te krijgen voor wat een padintegraal eigenlijk is en hoe Feynman op dit opmerkelijke idee is gekomen bekijken we in deze sectie enkele voorbeelden. We beginnen met een heuristische afleiding van de padintegraal en vervolgens bekijken we een voorbeeld van een berekening. Daarna geven we een concrete afleiding van de Feynman padintegraal die we in de rest van dit document zullen gebruiken.

\subsection{Conceptuele afleiding padintegraal}

Maar hoe moeten we dit conceptueel opvatten? We hebben gezien dat de fase $\phi$ kan worden uitgedrukt in de klassieke Lagrangiaan. Deze fase is complex en kunnen we ons voorstellen als een eenheidsvector in het complexe vlak onder hoek $\phi$ met de $x$-as. Dit noemen we een \emph{fasor}.

>>> $\phi$ is afhankelijk van de tijd en dus draait onze fasor rond met snelheid $\tdiff{\phi}{t}$ die grofweg gelijk is aan $L/\hslash$ (zie \see[sec:heurpad] hier boven).

Met de \emph{som over alle paden} doelt Feynman niet op het idee om alle paden bij elkaar op te tellen, maar om de fases $\phi_i$ behorende bij elk pad $p_i$ te sommeren. Dit kunnen we ons als volgt voorstellen. In \see[fig:fasoren] is ook goed te zien dat paden die verder van het klassieke pad $p_7$ liggen een grotere afwijking hebben. Daardoor hebben zij de neiging elkaar uit te doven.

\stopcomponent

% vim: spell spelllang=nl cole=1


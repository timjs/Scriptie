\documentclass[
  compress,
  handout,
  %draft,
]
{beamer}

%\usepackage[utf8]{inputenc}
%\usepackage[T1]{fontenc}
%\usepackage{lmodern}

\usepackage{fontspec}
\defaultfontfeatures
  {Scale=MatchLowercase,
   Ligatures={TeX}}
%\usepackage{mathspec}
\setmainfont{Aldus LT Std}
\setsansfont{Optima LT Std}
%\setmathfont{Asana Math}

\let\phi\varphi
\let\epsilon\varepsilon

\usepackage{calc}
%\usepackage[dutch]{babel}
\usepackage{polyglossia}
\setdefaultlanguage{dutch}

\usepackage{tikz}
%\usetikzlibrary{spy}
\input tikzstyles

\newcommand{\e}{
  \ensuremath{\mathrm{e}}
}
\renewcommand{\d}{
  \ensuremath{\;\text{d}}
}
\newcommand{\D}{
  \ensuremath{\;\mathcal{D}}
}
\newenvironment{twocolumns}[1][0.5]{
  \begin{columns}
    \begin{column}{#1\textwidth}
      \newcommand{\nextcolumn}[1][]{
      \end{column}
      \begin{column}[##1]{\textwidth-#1\textwidth}
      }
    }{
    \end{column}
  \end{columns}
}
\renewenvironment<>{figure}{
  \begin{actionenv}#1
    \begin{originalfigure}
    }{
    \end{originalfigure}
  \end{actionenv}
}

%\setbeameroption{show notes}

% Outer:
\useoutertheme{smoothbars}%miniframes,smoothbars
\usecolortheme{whale}%whale,seahorse,dolphin

% Inner:
\useinnertheme{rectangles}%rectangles,circles
\usecolortheme{rose}%lily,orchid,rose

% Adjustments:
%\definecolor{darkred}{RGB}{163,0,0}%dark red beaver
%\definecolor{darkred}{RGB}{174,59,39}%preview ru
%\definecolor{darkred}{rgb}{0.68,0.23,0.15}
%\definecolor{darkred}{RGB}{190,50,25}%eigen waarde ru
%\definecolor{darkred}{rgb}{0.75,0.20,0.10}
%\definecolor{darkred}{RGB}{158,45,29}%algemene waarde ru
%\definecolor{darkred}{rgb}{0.62,0.18,0.11}
%\definecolor{darkorange}{RGB}{177,98,30}

\usecolortheme[named={orange}]{structure}
\setbeamercolor{alerted text}{fg=structure}
%\setbeamercolor{section in head/foot}{fg=white,bg=black}
%\setbeamercolor{frametitle}{fg=structure,bg=white}
%\setbeamercolor{subsection in head/foot}{fg=white,bg=orange}
%\setbeamercolor{subsection in head/foot}{fg=white,bg=darkorange}
%\usecolortheme{beaver}

\everymath{\displaystyle}

\AtBeginSection[]{
  \begin{frame}{Plan}
    \tableofcontents[currentsection,hideothersubsections]
  \end{frame}
}

\title{Oneindig veel paden en complexe acties}
\subtitle{Bachelorstage bij Theoretische Hoge Energiefysica}
\author{Tim Steenvoorden}
\institute{Onder begeleiding van Wim Beenakker}
\date{20 juni 2012}
\titlegraphic{\includegraphics[width=0.1\textwidth]{afbeeldingen/ru-logo}}

\begin{document}

\begin{frame}
  \titlepage
\end{frame}

\section*{Introductie}
\subsection*{Wat gaan we doen?}

\begin{frame}{Er was eens\dots}

  Een gerenommeerd theoretisch natuurkundige die

  \begin{itemize}
    \item veel heeft bijgedragen aan de \emph{Snaartheorie}
    \item de \emph{Humboldt-prijs} heeft gewonnen
    \item leken enthousiasmeert voor het vak
  \end{itemize}

\end{frame}

\begin{frame}{Zijn naam}
  \begin{twocolumns}

    \begin{block}{Holger Bech Nielsen}
      Professor aan het Niels Bohr-instituut in Kopenhagen
    \end{block}

    \nextcolumn

    \includegraphics[width=\columnwidth]{afbeeldingen/holger_bech_nielsen}

  \end{twocolumns}
\end{frame}

\begin{frame}{Controversieel idee}

  \begin{quote}
    Wat gebeurt er met de natuurkunde wanneer we een imaginair deel toevoegen aan de actie?
  \end{quote}

  \pause
  \begin{itemize}
    \item ``Dingen die nu gebeuren worden niet alleen beïnvloed door het verleden, maar ook door de toekomst.''
    \item ``We krijgen het Higgsdeeltje nooit te zien, dit heeft het deeltje in de toekomst al besloten.''
    %\item ``We kunnen net zo goed een simpel kansexperiment doen met een pakje kaarten, om te besluiten of we de LHC moeten sluiten.''
    \item \dots
  \end{itemize}

  \pause
  \bigskip
  Een geniaal idee of een beetje over de top?

\end{frame}

\section*{}

\begin{frame}{Plan}
  \tableofcontents[hideallsubsections]
\end{frame}

\section[Idee]{Is dit een aanvaardbaar idee?}
\subsection*{Is dit een aanvaardbaar idee?}

\begin{frame}{Valt wel mee}

  \begin{itemize}
    \item Toevoeging aan bestaande wetten
    \item Ook gebeurt van klassiek naar kwantum
    \item \emph{Zolang ze in de klassieke limiet geen invloed hebben!}
  \end{itemize}

  \bigskip

  \pause
  \begin{twocolumns}

    \begin{block}{De Broglie golflengte}
      $ \lambda = \frac{h}{p} $
    \end{block}

    \nextcolumn

    Klein ten opzichte van object

    $\implies$

    kwantum effecten verwaarloosbaar.

  \end{twocolumns}

\end{frame}

\section[Methoden]{Met welke rekenmethoden kunnen we dit testen?}
\subsection*{Met welke rekenmethoden kunnen we dit testen?}

\begin{frame}{Zoektocht}

  \begin{itemize}
    \item Maakt gebruik van de \emph{klassieke actie}:\\
      $ S = \int_{t_A}^{t_B} L(x,\dot{x},t) \d t $
    \item Legt link tussen actie en \emph{kwantummechanica}
    \item Doet iets met \emph{complexe getallen}
  \end{itemize}

  \bigskip

  \pause
  \begin{twocolumns}

    \begin{block}{Feynman padintegraal}
      $ K(B,A) = \int_{x_A}^{x_B} \exp[ i S / \hslash ] \D x(t) $
    \end{block}

    \nextcolumn

    Integraal over \emph{alle mogelijke paden} van $A$ naar $B$.

  \end{twocolumns}

\end{frame}

\section[Werking]{Hoe werken deze methoden ook al weer?}
\subsection*{Hoe werken deze methoden ook al weer?}

\begin{frame}{Twee spleten experiment}

  \pause
  \begin{figure}
    \begin{tikzpicture}[scale=0.6]
      \coordinate[source,label=left:$B$] (B) at (0,5);
      \pause
      \coordinate (D) at (10,5);
      \draw[screen lines] (10,1) -- (10,9) node[above] {$D$};

      \pause
      \coordinate (S1) at (5,7);
      \coordinate (S2) at (5,3);
      \draw[screen lines] (5,1) -- (S1) -- (S2) -- (5,9) node[above] {$s$};
      \pause
      \fill[slit] (S1) circle;
      \fill[slit] (S2) circle;

      \pause
      \draw[just lines] (B) -- (S1) -- (D);
      \pause
      \draw[just lines] (B) -- (S2) -- (D);
    \end{tikzpicture}
  \end{figure}

\end{frame}

% Overlays!
\begin{frame}{Interferentie}
  \begin{twocolumns}

    \begin{figure}
      \begin{tikzpicture}[scale=0.4]
        \coordinate[source,label=left:$B$] (B) at (0,5);
        \coordinate (D) at (10,5);
        \draw[screen lines] (10,1) -- (10,9) node[above] {$D$};

        \coordinate (S1) at (5,7);
        \coordinate (S2) at (5,3);
        \draw[screen lines] (5,1) -- (S1) -- (S2) -- (5,9) node[above] {$s$};

        \fill<2,3,6->[slit] (S1) circle;
        \draw<2,3,6->[just lines] (B) -- node[auto,number] {1} (S1) -- (D);

        \fill<4,5,6->[slit] (S2) circle;
        \draw<4,5,6->[just lines] (B) -- node[auto,swap,number] {2} (S2) -- (D);
      \end{tikzpicture}
    \end{figure}

    \nextcolumn

    \begin{figure}
      \begin{tikzpicture}[xscale=0.5,just lines,domain=-4.5:4.5]
        \useasboundingbox (-5,-1.2) rectangle (5,2.2);
        \draw<3->[yshift=50] plot (\x,{\Gauss{-2}{1}}) node[right,number] {1};
        \draw<5->[yshift=20] plot (\x,{\Gauss{ 2}{1}}) node[right,number] {2};
        \draw<6->[help lines] (-5,0) -- (5,0) node[right] {$+$};
        \draw<6->[yshift=-30] plot (\x,{\Gauss{-2}{1} + \Gauss{2}{1}});
        %\draw<7->[heavy lines] (-5,-1) -- (4.5,2);
      \end{tikzpicture}
    \end{figure}

  \end{twocolumns}
  \begin{twocolumns}

    \begin{block}<8->{Kansen niet optellen!}
      $ P \neq P_1 + P_2 $
    \end{block}

    \nextcolumn

    \begin{figure}<7->
      $\neq$

      \includegraphics[width=\columnwidth]{afbeeldingen/inference}
    \end{figure}

  \end{twocolumns}
\end{frame}

\begin{frame}{Postulaten}

  \note{Hiervoor introduceert Feynman drie postulaten waarmee hij de kwantummechanica opbouwt.}

  \begin{twocolumns}[0.3]
    \begin{block}{Postulaat 1}
      $ P = |\psi|^2 $
    \end{block}
    \nextcolumn
    $\psi$ is de complexe \emph{waarschijnlijkheidsamplitude}
  \end{twocolumns}

  \pause
  \begin{twocolumns}[0.3]
    \begin{block}{Postulaat 2}
      $ \psi = \psi_1 + \psi_2 $
    \end{block}
    \nextcolumn
    die we \emph{wel} bij elkaar mogen optellen
  \end{twocolumns}

  \pause
  \begin{twocolumns}[0.3]
    \begin{block}{Postulaat 3}
      $ \psi_n \sim \exp [i S_n / \hslash] $
    \end{block}
    \nextcolumn
    en wordt berekend met de actie! 
  \end{twocolumns}

\end{frame}

\section[Berekening]{Hoe ziet zo'n berekening er uit?}
\subsection*{Hoe ziet zo'n berekening er uit?}

\begin{frame}{Reflecties}

  \begin{figure}
    \begin{tikzpicture}[scale=0.5]
      \useasboundingbox (-3,-1) rectangle (13,10);
      \coordinate<2->[source,label=above:$B$] (B) at ( 0,10);
      \coordinate<3->[detector,label=above:$D$] (D) at (10,10);

      \draw<4->[ultra thick] (-3,0) -- (13,0) node[right] {$r$};

      \draw<5,6>[thin,orange] (B) -- (5,0) node[below] {\alert{$p$}} -- (D);
      \foreach \x [count=\l from 0] in {-2,...,12}
        \draw<7->[thin,light orange] (B) -- (\x,0) node[below] {$p_{\l}$} -- (D);
      \draw<7->[thin,orange] (B) -- (5,0) node[below] {\alert{$p_7$}} -- (D);
    \end{tikzpicture}

    \visible<6->{Alle paden mogen meedoen, \emph{ook} paden die klassiek onmogelijk zijn.}
  \end{figure}

\end{frame}

% Overlays!
\begin{frame}{Fases en acties}
  \begin{twocolumns}[0.7]

    \begin{itemize}
      \item<1-> Voor elk pad klassieke actie uitrekenen
      \item<3-> Geeft ons een \emph{fasor}
        $ \psi_n \sim \exp [i \alert{S / \hslash} ] \equiv \exp [i \alert{\phi} ] $
    \end{itemize}

    \nextcolumn

    \begin{figure}<4->
      \begin{tikzpicture}[scale=0.7]
        \draw[axis lines] (-0.2,0) -- (2.2,0) node[right] {$\Re$};
        \draw[axis lines] (0,-0.2) -- (0,2.2) node[above] {$\Im$};

        \draw[->] (0,0) -- node[auto] {$1$} (30:2) node[right] {$\psi_n$};
        \draw[just lines] (1,0) arc[start angle=0,end angle=30,radius=1] node[below right,orange] {$\phi$};
      \end{tikzpicture}
    \end{figure}

  \end{twocolumns}

  \begin{figure}<2->
    \begin{tikzpicture}[scale=0.5]
      \draw[help lines] (-8,0) grid (7.5,6.5);

      \draw[axis lines] (-8.5,0) -- (8,0);
      \draw[axis lines] (-8,-0.5) -- (-8,7) node[below left] {$S$};

      \foreach \x [count=\l from 0] in {-7,...,7}
        \node[below] at (\x,0) {$p_{\l}$};
      \node[below] at (0,0) {\alert{$p_7$}};

      \draw[light lines] plot[samples at={-7,...,7},smooth,mark=*,mark options={orange}] (\x,{\x*\x/10+1});
    \end{tikzpicture}
  \end{figure}

  \note{We zien heel mooi dat de actie bij het klassieke pad $p_7$ minimaal is.}

\end{frame}

% Overlays!
\begin{frame}{Som over paden}
  \begin{twocolumns}[0.7]

  \begin{figure}
    \begin{tikzpicture}
      \fill<9>[background orange] (3,1) circle[radius=1.5];
      \fill<10>[background orange] (0.5,0) circle[radius=1];

      \draw[help lines] (-0.9,-0.9) grid (6.9,3.9);

      \draw[axis lines] (-1,0) -- (7,0) node[right] {$\Re$};
      \draw[axis lines] (0,-1) -- (0,4) node[above] {$\Im$};

      \path (0,0) coordinate (P0)
      \foreach \a [count=\i from 1] in {355.05, 241.74, 153.24, 87.55, 41.51, 11.62, 354.94, 349.59, 354.94, 11.62, 41.51, 87.55, 153.24, 241.74, 355.05}
        { ++(\a:1) coordinate (P\i)};

      \path<7-> (P15) node[above] {$D$};
      \draw<7->[heavy lines,->] (P0) -- (P15);

      \path<2-> (P0) node[below right] {$B$};
      \draw<3->[just lines,->] (P0) -- node {$\psi_{0}$} (P1);
      \draw<4->[just lines,->] (P1) -- node {$\psi_{1}$} (P2);
      \draw<5->[just lines,->] (P2) -- node {$\psi_{2}$} (P3);
      \foreach \to [remember=\to as \from (initially 3)] in {4,...,15}
        \draw<6->[just lines,->] (P\from) -- node {$\psi_{\from}$} (P\to);
    \end{tikzpicture}
  \end{figure}

  \nextcolumn

  \begin{itemize}
    \item<8-> Lengte resulterende vector geeft kans
    \item<9-> Dichtbij klassieke pad: fase varieert weinig
    \item<10-> Totaal niet klassieke paden: fases variëren snel% ten opzichte van elkaar
  \end{itemize}

  \end{twocolumns}
\end{frame}

\begin{frame}{Imaginair deel}
  \begin{twocolumns}

    \begin{align*}
      S &\mapsto S_r + iS_i \\
      \phi &\mapsto \phi_r + i\phi_i
    \end{align*}

    \pause
    \nextcolumn

    \begin{figure}
      \begin{tikzpicture}[scale=0.7]
        \draw[axis lines] (-0.2,0) -- (2.2,0) node[right] {$\Re$};
        \draw[axis lines] (0,-0.2) -- (0,2.2) node[above] {$\Im$};

        \draw[->] (0,0) -- node[auto] {??} (60:3) node[right] {$\psi_n$};
        \draw[just lines] (1,0) arc[start angle=0,end angle=60,radius=1] node[right,orange] {??};
      \end{tikzpicture}
    \end{figure}

  \end{twocolumns}

  Veranderingen aan fasor $\psi$:

  \begin{itemize}
    \item hoek
    \item lengte\\
      (``absorptie'' of ``impact'' door reflecterende plaat)
  \end{itemize}

  \bigskip

  Simulatie om effecten te onderzoeken

\end{frame}

\section*{Conclusie}
\subsection*{Wat hebben we gezien?}

\begin{frame}{(Voorlopige) antwoorden}

  \begin{enumerate}
    \item Aanvaardbaar om bestaande wetten uit te breiden
    \item Testen met padintegralen
    \item Oplossen door sommeren van fasoren
    \item Complexe actie verandert lengte en hoek fasor
  \end{enumerate}

\end{frame}

\end{document}

\begin{frame}{Details}

  \begin{tikzpicture}[
      %scale=0.5,
      spy using overlays={circle,magnification=2,size=2cm,connect spies}
    ]
    \draw[help lines] (-0.9,-0.9) grid (6.9,3.9);

    \draw[axis lines] (-1,0) -- (7,0) node[right] {$\Re$};
    \draw[axis lines] (0,-1) -- (0,4) node[above] {$\Im$};

    \path (0,0) coordinate (P0)
    \foreach \a [count=\i from 1] in {355.05, 241.74, 153.24, 87.55, 41.51, 11.62, 354.94, 349.59, 354.94, 11.62, 41.51, 87.55, 153.24, 241.74, 355.05}
    { ++(\a:1) coordinate (P\i)};

    \path (P0) node[below right] {$B$};
    \foreach \to [remember=\to as \from (initially 0)] in {1,...,15}
    \draw[light lines,->] (P\from) -- node {$p_{\from}$} (P\to);
    \path (P15) node[above] {$D$};

    \draw[heavy lines,->] (P0) -- (P15);

    \spy on (P7) in node at (10,3);
    \spy on (P0) in node at (10,-1);
  \end{tikzpicture}

  \begin{itemize}
    \item Dicht bij klassiek pad varieert de fase weinig
    \item Totaal niet klassieke paden variëren snel ten opzichte van elkaar
  \end{itemize}

\end{frame}

% vim: syn=latex spell spl=nl

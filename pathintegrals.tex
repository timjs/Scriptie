\startcomponent pathintgrals
\environment thesislayout

\chapter{Feynman Padintegralen}

\todo{Titels aanpassen}

\startemphasize
\todo{Inleiding}
\stopemphasize

\section{}

\startitemize
\item \todo{Wat zijn padintegralen?}
\item \todo{Wat doen ze?}
\item \todo{Hoe werken ze?}
\item \todo{Waar zijn ze goed voor?}
\stopitemize

\section{}

\startitemize
\item \todo{Hoe kunnen we ze uitrekenen?}
\item \todo{Welke methoden zijn er?}
\stopitemize

\section{}

\todo{Welke klasse kunnen we exact uitrekenen?}

In de vorige sectie hebben we kunnen zien dat een padintegraal met $V \approx x$ eenvoudig exact op te lossen is. Lang niet alle padintegralen zijn analytisch oplosbaar. Welke restrictie moeten we opleggen? Aangezien we de kinetische-energie term van de Lagrangiaan niet kunnen aanpassen, kunnen we deze vraag reduceren tot: waar moet de potentiaal $V$ aan voldoen?

Laten we nogmaals de padintegraal uitrekenen voor één deeltje met spin $0$ en massa $m$, maar nu met een algemene potentiaal

\startformula
V(X).
\stopformula

We passen dezelfde truc toe als in \see[sec:gravitatieveld] waarbij we $X(T)$ opsplitsen in een klassiek pad $r(T)$ en quantumfluctuaties $q(T)$

\startformula
X(T)  :=  r(T) + q(T)
\stopformula

De vorm van $r(T)$ en $q(T)$ doen er nu even niet toe. We vinden voor de Lagrangiaan

\todo{\tex{wall}?}

\startalignedformula
L (X, \dot{X})
&=  \half m \dot{X}^2 - V(X)  \\
&=  \half m (\dot{r} + \dot{q})^2 - V(r + q)  \\
&=  \half m \dot{r}^2 + \half m \dot{q}^2 + m \dot{r} \dot{q} - V(r + q)  \\
\stopalignedformula

Waarin we de eerste drie termen herkennen uit \see[for:LagrangiaanVoorGravitatieVeld]. Nu kunnen we helaas niet meteen het klassieke deel en het quantum deel van elkaar scheiden. Maar voor onze klassieke Lagrangiaan

\startformula
L_r (r, \dot{r})  :=  \half m \dot{r}^2 - V(r),
\stopformula

moet de Euler-Lagrange vergelijking gelden.

Dit houden we in ons achterhoofd. Wanneer we de klassieke Lagrangiaan invullen in \see[for:LagrangiaanVoorAlgemenePotentiaal] lopen we een extra $V(r)$ op. Daarnaast kunnen we $\dot{r} \dot{q}$ vervangen door

\startformula
\dd{T} \dot{r} q  =  \ddot{r} q + \dot{r} \dot{q}
\stopformula

\startformula
L (X, \dot{X})  =  L_r + V(r) + \half m \dot{q}^2 + m (\dd{T} \dot{r} q - \ddot{r} q) - V(r + q)
\stopformula

Door de Dirichlet voorwaarde valt $\dd{T} \dot{r} q$ weg wanneer we overstappen op de actie (zie \see[sec:DirichletVoorwaarde]). De laatste manipulatie volgt uit de Euler-Lagrange vergelijking op de klassieke Lagrangiaan.

\startalignedformula
\dd{T} \pdiff{L_r}{\dot{r}}  &=  \pdiff{L_r}{r}  \\
\dd{T} m \dot{r}             &=  - \tdiff{V}{r}  \\
m \ddot{r}                   &=  - \tdiff{V}{r}  \\
\stopalignedformula

We herkennen $m \ddot{r}$ en we krijgen

\startformula
L (X, \dot{X})  =  L_r + V(r) + \half m \dot{q}^2 - \tdiff{V}{r} q - V(r + q)
\stopformula

Laten we eens goed naar dit resultaat kijken. Dit is onze Lagrangiaan voor een deeltje in een dimensie. We hebben het pad $X(T)$ opgesplitst in een klassiek- en een quantum deel. Het doel is om onze Lagrangiaan ook zo op te splitsen. Stiekem hebben we dit al een beetje gedaan: de klassieke Lagrangiaan $L_r$ staat er al in. Dat betekent dat alle andere termen samen het quantumdeel moeten vormen. Maar sommige zijn nog afhankelijk van $r$!

Dat is dus de eis die we moeten opleggen aan de potentiaal. Het volgende moet onafhankelijk zijn van $r$.

\startformula
\half m \dot{q}^2 - \tdiff{V}{r} q - V(r) + V(r + q)
\stopformula

Met andere woorden

\startalignedformula
\dd{r} [ - \tdiff{V}{r} q - V(r) + V(r + q) ]            &=  0  \\
\tdiff{V^2}{^2 r} q + \tdiff{V}{r} - \tdiff{V}{(r + q)}  &=  0  \\
\stopalignedformula

Hieraan voldoen precies de tweedegraads polynomen in $X$.

\startalignedformula
V(r + q)           &=  a + b (r + q) + c (r + q)^2  \\
%\tdiff{V}{r}       &=  b + 2 c (r + q)  \\
%\tdiff{V^2}{^2 r}  &=  2 c  \\
\stopalignedformula

Zodat

\startformula
\tdiff{V^2}{^2 r} q + \tdiff{V}{r} - \tdiff{V}{(r + q)}
=  2 c q + b + 2 c (r + q) - b - 2 c (r + q)
=  0
\stopformula

\section{}

\todo{Hier pas rekenmethode met Fourierdecompositie}

\section{}

\todo{Hoe reduceren (benaderen) we een andere klasse tot bovenstaande?}

\stopcomponent

% vim: spell spelllang=nl


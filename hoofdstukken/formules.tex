\startcomponent formules
\product        scriptie
\environment    thesislayout

\chapter[app:formules]{Gebruikte standaardfuncties}

\section[sec:fresnelintegralen]{Fresnelintegralen}

Bij de berekeningen in hoofdstuk \see[chp:afleiding] maken we gebruik van \emph{Fesnelintegralen}. Dit zijn de functies $\fres[z]$ en $\frec[z]$. Ze zijn gedefinieerd als een integraal over respectievelijk de sinus en de cosinus van $\halfpi t^2$. De integratiegrenzen lopen van nul tot aan het argument $z$.
  \placeformula[for:fresnel definitie]
  \startformula\startspread
  \fres[z] := \int_0^z \sin\left[ \halfpi t^2 \right] \d t \SC
  \frec[z] := \int_0^z \cos\left[ \halfpi t^2 \right] \d t.
  \stopspread\stopformula
Door hun definitie als integraal, erven ze ook de eigenschappen hiervan. Zo kunnen we een min-teken uit het argument trekken.
  \placeformula[for:fresnel min eigenschap]
  \startformula\startspread
  \fres[-z] = -\fres[z] \SC
  \frec[-z] = -\frec[z]
  \stopspread\stopformula
Als argument kunnen we ook een complex getal opgeven. In het geval dat het argument volledig imaginair is, kunnen we de complexe eenheid $i$ uit het argument trekken. In het geval van de sinus levert ons dat een \emph{extra min} op.
  \placeformula[for:fresnel complex eigenschap]
  \startformula\startspread
  \fres[i z] = - i \fres[z] \SC
  \frec[i z] =   i \frec[z]
  \stopspread\stopformula
Verder kunnen we in \see[fig:fresnelintegralen] zien dat beide functies in het oneindige convergeren naar een half.
  \placeformula[for:fresnel limiet eigenschap]
  \startformula\startspread
  \fres[\infty] = \half \SC
  \frec[\infty] =   \half \frec[z]
  \stopspread\stopformula

\placefigure[][fig:fresnelintegralen]
  {Fresnelintegralen, links de sinus, rechts de cosinus.}
  \startcombination[2*1]
  {\externalfigure[fres][width=0.5\textwidth]} {$\fres[z]$}
  {\externalfigure[frec][width=0.5\textwidth]} {$\frec[z]$}
  \stopcombination

\page%FIXME

Fresnelintegralen kunnen we gebruiken bij het integreren van niet triviale exponenten. Zo komen we in \see[sec:integreren] een kwadratische functie tegen in een exponent. Deze Gaussische functie is op te lossen met de identiteit
  \placeformula[for:exponent identiteit]
  \startformula\startsplit
  \int \exp[\alpha z^2 + \beta z + \gamma] \d z \SC
  = \frac{1-i}{2} \sqrt{\frac{\pi}{\alpha}} \exp\left[ \gamma - \frac{\beta^2}{4\alpha} \right] \SR
  \cdot \left( \frec\left[ \frac{1+i}{2} \frac{\beta + 2 \alpha z}{\sqrt{\pi \alpha}} \right]
  - i \fres\left[ \frac{1+i}{2} \frac{\beta + 2 \alpha z}{\sqrt{\pi \alpha}} \right] \right).
  \stopsplit\stopformula
Wanneer we bovenstaande willen uitrekenen met grenzen $\pm\infty$ wordt de integraal gegeven door
  \placeformula[for:exponent identiteit oneindig]
  \startformula
  \int_{-\infty}^\infty \exp[\alpha z^2 + \beta z + \gamma] \d z
  = \frac{\sqrt{\pi}}{2\sqrt{\alpha}} \exp\left[ \gamma - \frac{\beta^2}{4\alpha} \right].
  \stopformula

\section[sec:dirac delta]{Dirac deltafunctie}

De \emph{een-dimensionale Dirac deltafunctie} $\delta(x)$ is een oneindig hoge, oneindig dunne piek met oppervlakte één. We definiëren hem stuksgewijs.
  \placeformula[for:dirac delta]
  \startformula
  \delta(x) := \startcases
    \NC \infty \MC \text{als } x = 0 \NR
    \NC 0 \MC \text{als } x \neq 0 \NR
  \stopcases\stopformula
Met extra kanttekening dat wanneer we over de deltafunctie integreren, er één uit moet komen.
  \placeformula[for:dirac integral]
  \startformula
  \int_{-\infty}^{+\infty} \delta(x) \d x = 1
  \stopformula

\placefigure[][fig:dirac delta]
  {Dirac deltadistributie}
  {\externalfigure[deltapeak][width=0.5\textwidth]}

Het product van een nette functie $f(x)$ en de deltafunctie $\delta(x)$ levert overal nul op, \emph{behalve} op $x=0$. We \emph{selecteren} als het ware de waarde van $f(x)$ op $x=0$. Met behulp van bovenstaande definitie geldt
  \startformula
  \int_{-\infty}^{+\infty} f(x) \delta(x) \d x = f(0) \int_{-\infty}^{+\infty} \delta(x) \d x = f(0).
  \stopformula
Hierbij hoeven de grenzen niet van $-\infty$ tot $+\infty$ te lopen. Een klein gebeid rond $x=0$ is al voldoende.
Bovenstaande kunnen we generaliseren tot
  \placeformula[for:diract delta algemeen]
  \startformula\startspread[m=3]
  \delta(x-a) := \startcases
    \NC \infty \MC \text{als } x = a \NR
    \NC 0 \MC \text{als } x \neq a \NR
  \stopcases \SC
  \text{waarbij} \SC
  \int_{-\infty}^{+\infty} \delta(x-a) \d x = 1
  \stopspread\stopformula

Er zijn verschillende integralen die een Dirac deltafunctie opleveren. Een waarvan wij gebruik maken is
  \placeformula[for:delta indentiteit]
  \startformula
  \int_0^\infty \exp[i x z] \d z = \pi \delta(x).
  \stopformula

\stopcomponent

% vim: ft=context spell spl=nl cole=1

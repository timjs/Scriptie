\startcomponent code
\product        scriptie
\environment    thesislayout

\chapter{Mathematicacode}

\startemphasize
De methode van \see[chp:reflecties] is uitgewerkt als programma in Wolfram Mathematica. Hieronder is de code van dit programma te vinden. Door te spelen met de waarden voor \type{plateLength} en \type{numberOfPaths} zijn figuren \in[fig:simulatie aantal] en \in[fig:simulatie massa] geproduceerd.
\stopemphasize

\startlinenumbering
\typefile{simulaties/simulatie-reel.m}
\stoplinenumbering

%\typemathematicafile{simulaties/simulatie-reel.m}

\stopcomponent

Extra spaties:
\startitemize
\item =
\item ->
\item +
\item //
\item /@ %@
\item &
\item , 
\stopitemize

% vim: ft=context spell spl=nl cole=1

\startcomponent conlcusie
\product        scriptie
\environment    thesislayout

\chapter[chp:conclusie]{Conclusie}

De padintegraal wordt vaak gezien als conceptueel lastig en wiskundig abstract. We hebben het in \see[chp:reflecties] voor elkaar gekregen om, met behulp van simpele reflecties aan een plaat, de belangrijkste eigenschappen van de padintegraal aanschouwelijk te maken. Allereerst het uitrekenen van de afzonderlijke \emph{bijdragen}, zoals het derde postulaat van Feynman aangeeft. Deze kunnen we representeren met \emph{fasoren}. Vervolgens het optellen van deze bijdragen. Dit is equivalent met het optellen van fasoren in het complexe vlak. De figuren die daarbij ontstaan geven inzicht in de begrippen \emph{interferentie}, constructief, dan wel destructief, \emph{nabij klassiek} en \emph{ver van klassiek} en de \emph{waarschijnlijkheidsamplitude} zelf. Kortom, we hebben er voor gezorgd dat we de wiskunde kunnen uitbeelden en dat de lezer zich iets kan voorstellen bij een padintegraal.

Het sluitend maken van deze formulering is helaas niet gelukt. Het normaliseren van de waarschijnlijkheidsamplitude die uit een reflectie-experiment komt rollen bleek lastig. Dat hebben we kunnen zien in \see[chp:normalisatie]. Een één dimensionale methode blijkt niet mogelijk te zijn. Wanneer we een normalisatieconstante willen uitrekenen met dezelfde methode waarmee we in \see[chp:reflecties] te werk zijn gegaan, blijkt de resulterende integraal analytisch ondoenlijk te zijn.
%In onze zoektocht naar een alternatief, hebben we een aanname gemaakt die niet mogelijk is binnen het kader van \see[chp:reflecties].
Bij het uitwerken van een alternatief, hebben we een aanname gemaakt die niet strookt met het experiment uit \see[chp:reflecties].

Als het ons zou lukken om de normalisatie sluitend te krijgen, zouden we dit eenvoudige experiment verder kunnen uitbuiten om eigenschappen van padintegralen mee te testen, te visualiseren en te doorgronden. Helaas speelt tijd ook een rol, en hierdoor is er nog genoeg om uit te zoeken als vervolg op deze scriptie.

\stopcomponent


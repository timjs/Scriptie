\startcomponent paden
\product        scriptie
\environment    thesislayout

\chapter[chp:paden]{Oneindig veel paden}

\startemphasize
De meeste teksten introduceren de kwantummechanica aan de hand van canonieke kwantisatie en de bijbehorende operatoralgebra van Dirac en Von Neumann. In dit eerste hoofdstuk bekijken wij echter de veel paden formulering van de kwantummechanica. Deze alternatieve methode levert ons niet alleen een krachtige rekenmethode op, maar ook een heel andere kijk op de voortbeweging van kwantumdeeltjes.

Aan de hand van een tweetal experimenten bekijken we de verschillen tussen het klassieke pad en de kwantummechanische paden die een deeltje kan afleggen. Hierbij komt onder andere de relatie tussen waarschijnlijkheid en fase ter sprake. Door dit experiment stap voor stap uit te breiden komen we uiteindelijk bij de padintegraal van Feynman. Dit is de basis voor de overige hoofdstukken.
\stopemphasize

\section[sec:klassiek pad]{Het klassieke pad}

Het eerste experiment dat wij in dit hoofdstuk beschouwen is simpel van opzet. We nemen een bron $B$ van deeltjes met massa $m$. Op afstand $x$ zetten we een detector $D$ die de uitgezonden deeltjes opvangt. Dit is weergegeven in \see[fig:simpel experiment]. Voor nu is het voldoende om klassieke deeltjes door $B$ te laten produceren. Later zullen we juist naar kwantummechanische deeltjes gaan kijken. Hiervoor zullen we in de rest van dit hoofdstuk dit experiment stap voor stap uitbreiden.

\placefigure[][fig:simpel experiment]
  {Eerste opzet van ons experiment. Een bron $B$ met op een afstand $x$ een detector $D$. Klassieke deeltjes reizen over het pad met de kleinste actie van $B$ naar $D$.}
  {\externaltikzpicture[simpel_experiment][scale=0.5]}

In de klassieke wereld is het niet ingewikkeld om het \emph{klassieke pad} $\bar{x}(t)$ uit te rekenen dat een deeltje van $B$ naar $D$ aflegt. Natuurlijk zijn de wetten van Newton een goed hulpmiddel, maar laten wij gebruik maken van de (meer analytische) formalismen van Lagrange \cite[Beenakker:2007uc,Goldstein:2002uj,Taylor:2005uj]. Lagrange karakteriseert een systeem met de vergelijking
  \startformula
  L := T - V,
  \stopformula
de naar hem vernoemde \emph{Lagrangiaan}. Hierbij zijn $T$ de kinetische energie van het systeem en $V$ de potentiaal waarin het systeem zich bevind. Op deze manier houden we rekening met zowel de interne eigenschappen van het systeem als met de invloeden van de omgeving. Voor een deeltje met kinetische energie $\half m \dot{x}^2$ bewegend in een potentiaal $V(x,t)$ variërend in plaats en tijd ontstaat
  \startformula
  L(x,\dot{x},t) = \half m \dot{x}^2 - V(x,t).
  \stopformula
Hier is $L$ expliciet een functie van $x$, $\dot{x}$ en $t$. We maken gebruik van de notatie $\dot{x} = \tdiff{x}{t} = v$.

Om nu het pad te berekenen dat ons deeltje van de bron naar de detector aflegt maken we gebruik van \emph{het principe van de kleinste actie}. Dit wil zeggen dat we een grootheid $S$ (de \emph{actie}) invoeren die we voor elk pad kunnen uitrekenen. Het pad dat het deeltje uiteindelijk aflegt is dat pad waarbij $S$ minimaal is. Met andere woorden: $S$ verandert niet in eerste orde wanneer het klassieke pad $\bar{x}(t)$ een beetje verandert.
\todo{Bovenstaande kwantitatief maken door voorbeeld met variatierekening uit te werken?}

We definiëren de actie als een integraal over de Lagrangiaan:
  \placeformula[for:actie]
  \startformula
  S := \int_{t_i}^{t_f} L(x,\dot{x},t) \d t.
  \stopformula
Hierbij is de actie $S$ zelf niet zozeer interessant, als wel de manier waarop we deze uitrekenen. Om $S$ te berekenen hebben we namelijk de naburige paden nodig om \emph{dat} pad uit te zoeken dat de kleinste actie heeft. Iets om in ons achterhoofd te houden.

bla
  \startformula
  \delta x(t_i) = \delta x(t_f) = 0
  \stopformula
bla
  \startformula
  \delta S = S(\bar{x} + \delta x) - S(\bar{x}) = 0
  \stopformula
bla
  \startformula\startsplit
  S(\bar{x} + \delta x) \SC
  = \int_{t_i}^{t_f} L(\dot{x} + \delta \dot{x}, x + \delta x, t) \d t \SR
  = \int_{t_i}^{t_f} \left( L(x, \dot{x}, t) + \delta \dot{x} \pdiff{L}{\dot{x}} + \delta x \pdiff{L}{x} \right) \d t \SR
  = S(x) + \int_{t_i}^{t_f} \left( \delta \dot{x} \pdiff{L}{\dot{x}} + \delta x \pdiff{L}{x} \right) \d t
  \stopsplit\stopformula
bla
  \startformula
  \delta S = \delta x \left. \pdiff{L}{\dot{x}} \right|_{t_i}^{t_f}
  - \int_{t_i}^{t_f} \delta x \left( \ddiff{t} \pdiff{L}{\dot{x}} - \pdiff{L}{x} \right) \d t
  \stopformula
bla
  \startformula
  \ddiff{t} \pdiff{L}{\dot{x}} - \pdiff{L}{x} = 0
  \stopformula
bla

\section[sec:kwantum pad]{Kwantummechanische paden}

In de vorige sectie produceerde de bron alleen klassieke deeltjes. Laten we eens kijken wat er gebeurt wanneer we een bron gebruiken die kwantummechanische deeltjes maakt, zoals elektronen.
%Dit kunnen we doen door bijvoorbeeld een spanning te zetten op een gloeidraad.
Wanneer we nu de detector aanzetten, merken we dat we niet meer alle deeltjes meten die door $B$ zijn uitgezonden.
%We passen de opstelling dusdanig aan dat we detector $D$ vervangen door een detectiescherm $d$. Hierop kunnen we, nadat we enige tijd elektronen hebben geproduceerd, een spreiding waarnemen zoals te zien in \see[fig:detectiescherm].
Om een poging te wagen de \quote{verloren} deeltjes terug te vinden passen we de detector dusdanig aan dat we $D$ in verticale richting kunnen bewegen over een afstand $y$ (zie \see[fig:detectie]). Nu kunnen we de \emph{intensiteit} $I$ verkrijgen door het signaal van de detector op verschillende hoogtes te integreren over de tijd. Wanneer we dit uitzetten tegen de hoogte $y$ ontstaat een patroon zoals weergegeven in \see[fig:spreiding].

\placefigure[][fig:detectie]
  {In plaats van klassieke deeltjes produceert bron $B$ nu elektronen. We passen detector $D$ zo aan dat deze verticaal kan bewegen over een afstand $y$.}
  {\externaltikzpicture[detectie][scale=0.5]}

\placefigure[][fig:spreiding]
  {Spreiding zoals gemeten door de detector na integratie over de tijd. \todo{Grafieken tekenen}}
  {\externaltikzpicture[spreiding][scale=0.5]}

Het patroon in \see[fig:spreiding] lijkt er op het diffractiepatroon van een golf door één spleet. Toch kunnen we te weten komen of we echt met deeltjes te maken hebben. Bij een zwakke elektronenbron geeft de detector af en toe een signaal, niet continu. Dit wijst er op dat we losse objecten meten. Ook is de sterkte van het signaal altijd even groot (we gaan uit van een zeer gevoelige detector). We meten dus telkens één elektron en niet een halve of een andere breuk.

We moeten concluderen dat we niet meer kunnen spreken van \emph{het} pad dat een deeltje aflegt: een elektron heeft de mogelijkheid om meerdere paden af te leggen. Daarnaast kunnen we niet meer bepalen op welke plek een elektron aan komt. We kunnen alleen spreken over de \emph{waarschijnlijkheid} $P$ waarmee we een elektron op hoogte $y$ kunnen detecteren.
Uit \see[fig:spreiding] blijkt immers dat het nog steeds het meest waarschijnlijk is om een deeltje recht tegenover de bron aan te treffen, maar er zijn ook deeltjes die verder weg van deze as terecht komen. De kans hierop wordt wel steeds kleiner naarmate we ons verder van deze as begeven.

\section[sec:twee spleten,sec:postulaten]{Twee spleten}

Om het de elektronen moeilijker te maken om direct van $B$ naar $D$ te reizen, plaatsen we tussen de bron en de detector een scherm $s$ met daarin twee spleten $S_1$ en $S_2$ zoals in \see[fig:twee spleten] is weergegeven. We hebben net al gezien dat een elektron over verschillende paden kan reizen. De vraag is nu of een elektron uit $B$ zal \quote{kiezen} om via $S_1$ naar $D$ te reizen, of via $S_2$.
Dit is het welbekende twee spleten experiment van Young. We zullen hier de uitkomsten kort herhalen en koppelen aan de postulaten waarmee Feynman de kwantummechanica opbouwt.

\placefigure[][fig:twee spleten]
  {Het twee spleten experiment. Tussen bron $B$ en detector $D$ plaatsen we een scherm met daarin twee spleten $S_1$ en $S_2$. Deeltjes uit $B$ kunnen twee paden afleggen om in $D$ aan te komen.}
  {\externaltikzpicture[twee_spleten][scale=0.5]}

De spleten $S_1$ en $S_2$ kunnen we beschouwen als afzonderlijke bronnen. De elektronen uit $S_1$ zullen een spreiding vertonen zoals we hebben gezien in \see[fig:spreiding]. Hetzelfde geld voor elektronen uit $S_2$. Wanneer we beide spreidingen optellen ontstaat een patroon zoals te zien in \see[fig:spreidingen]{a}. Klaar! Of toch niet\dots

\placefigure[][fig:spreidingen]
  {a. Spreiding door optellen van de losse spreidingen behorende bij spleet $S_1$ en spleet $S_2$. b. Spreiding zoals gemeten door de detector bij het twee spleten experiment. \todo{Grafieken tekenen}}
  \startcombination[2*1]
  {\externaltikzpicture[spreiding_dubbel][scale=0.5]}        {a}
  {\externaltikzpicture[spreiding_interferentie][scale=0.5]} {b}
  \stopcombination

Wanneer we het experiment uitvoeren en de intensiteit uitzetten tegen de hoogte blijkt er een veel ingewikkelder patroon te ontstaan (zie \see[fig:spreidingen]{b}). Het moge duidelijk zijn dat bovenstaande redenatie niet klopt. Blijkbaar kunnen we niet zomaar de waarschijnlijkheden van $S_1$ en $S_2$ bij elkaar optellen, met andere woorden:
  \startformula
  P \neq P_1 + P_2.
  \stopformula
In plaats van direct de waarschijnlijkheid te bekijken stellen we dat $P$ is uit te rekenen met de modulus kwadraat van een complex getal $\psi$, de \emph{waarschijnlijkheidsamplitude}:
  \placeformula[for:waarschijnlijkheidsamplitude]
  \startformula
  P = |\psi|^2.
  \stopformula
Dit geeft het eerste postulaat van Feynman weer.
  \startpostulate
  De waarschijnlijkheid van een gebeurtenis wordt gegeven door de modulus kwadraat van een complex getal, genaamd de \emph{waarschijnlijkheidsamplitude}.
  \stoppostulate
Voor de afzonderlijke waarschijnlijkheidsamplitudes van $S_1$ en $S_2$ geldt nog steeds
  \startformula\startspread[m=3]
  P_1 = |\psi_1|^2  \SC
  \text{en}   \SC
  P_2 = |\psi_2|^2.
  \stopspread\stopformula
Het verschil komt pas bij het uitrekenen van van de $\psi$ voor het \emph{totale} experiment. We stellen nu dat we de waarschijnlijkheidsamplitudes wel mogen sommeren:
  \placeformula[for:som waarschijnlijkheidsamplitudes]
  \startformula
  \psi = \psi_1 + \psi_2.
  \stopformula
Zodat voor de waarschijnlijkheid van het totale experiment geldt
  \startformula
  P = |\psi|^2 = |\psi_1 + \psi_2|^2.
  \stopformula
Dit levert ons het tweede postulaat.
  \startpostulate
  %De waarschijnlijkheidsamplitude wordt verkregen door het sommeren van de bijdragen van alle \emph{paden} tussen begin en eindpunt.
  De waarschijnlijkheidsamplitude van een gebeurtenis is de som van de \emph{bijdragen} behorende bij de verschillende paden tussen begin en eindpunt.
  \stoppostulate

%Het patroon in \see[fig:spreidingen]{b} komt ons echter bekend voor. Het is precies het interferentiepatroon van een golf uit $S_1$ met een golf uit $S_2$. De amplitude van een golf kunnen we het beste beschrijven met een complex getal $\psi$. %FIXME
%In plaats van direct de waarschijnlijkheid $P$ te bekijken voeren we een nieuw begrip in, de \emph{waarschijnlijkheidsamplitude}. Dit is een complex getal $\psi$ dat we koppelen aan de waarschijnlijkheid met de formule

%\section{Fases en acties}

%Het patroon in \see[fig:spreidingen]{b} komt ons bekend voor. Het is precies het interferentiepatroon van een golf uit $S_1$ met een golf uit $S_2$. Een golf kunnen we het beste beschrijven met een complexe $\e$-macht.

\placefigure[right][fig:fasor]
  {Een vector met lengte $1$ onder hoek $\phi$ in het complexe vlak. Dit komt overeen met een fasor $\e^{i \phi}$.}
  {\externaltikzpicture[fasor]}

Om inzicht te krijgen in wat de waarschijnlijkheidsamplitude eigenlijk is, maken we een uitstapje naar de golfmechanica van Schrödinger. De Schrödingervergelijking geeft ons een beschrijving van de beweging van een kwantummechanisch deeltje in de vorm van een golfvergelijking. De algemene oplossing van deze vergelijking is
  \placeformula[for:vlakke golf]
  \startformula
  \Psi(\vec{r}, t) = A \e^{i (\vec{p} \cdot \vec{r} - E t) / \hslash}.
  \stopformula
Dit is een \emph{vlakke golf} met amplitude $A$ en complexe fase
  \startformula
  \phi := (\vec{p} \cdot \vec{r} - E t) / \hslash.
  \stopformula
Uitdrukkingen van de vorm $\e^{i \phi}$ noemen we een \emph{fasor}. Een fasor kunnen we weergeven in het complexe vlak als een eenheidsvector onder een hoek $\phi$ (zie \see[fig:fasor]).
Wanneer we met een deeltje meereizen over zijn pad zal $\phi$ veranderen met snelheid:
%Wanneer $\vec{r}$ de plaats aangeeft van de top van deze golf, dan wordt de faseverandering op dat punt gegeven door:
  \placeformula[for:faseverandering]
  \startformula\startsplit
  \tdiff{\phi}{t}  \SC
  = (\vec{p} \cdot \vec{v} - E) / \hslash  \SR
  = (m \vec{v} \cdot \vec{v} - \half m v^2 - V) / \hslash  \SR
  = (\half m v^2 - V) / \hslash  \SR
  = L / \hslash.
  \stopsplit\stopformula
Het faseverschil tussen twee tijdstippen $t_1$ en $t_2$ wordt dan
  \startformula
  \phi_2 - \phi_1 = \int_{t_1}^{t_2} L(t) / \hslash \d t = S(t_2,t_1) / \hslash.
  \stopformula

Wat we hebben gedaan is het \emph{faseverschil} op twee tijdstippen uitdrukken in de bijbehorende (klassieke) actie gewogen met $\hslash$. Wanneer we dit antwoord combineren met de golf in \see[for:vlakke golf] komen we op het laatste postulaat van Feynman.
  \startpostulate
  De bijdrage van een pad aan de waarschijnlijkheidsamplitude is proportioneel met $\e^{i S / \hslash}$ waarbij de \emph{actie} $S$ wordt berekend over dit specifieke pad.
  \stoppostulate
%Wanneer we nu weer naar de golf in \see[for:vlakke golf] kijken, is het geen vreemde gedachte deze te herschrijven naar
  %\placeformula[for:vlakke golf met actie]
  %\startformula
  %\Psi(\vec{r}, t) = A \e^{i \int L(t) / \hslash \d t} = A \e^{i S(t) / \hslash}.
  %\stopformula

\section[sec:reflecties]{Reflecties}

\placefigure[][fig:reflecties]
  {Een reflectie-experiment. Bron $B$ schiet deeltjes af op een reflecterende plaat $r$, die vervolgens door detector $D$ worden geregistreerd. Er zijn vijftien mogelijke paden getekend waarop een deeltje van $B$ naar $D$ kan komen. Het benadrukte pad is het klassieke pad.}
  {\externaltikzpicture[reflecties][scale=0.5]}

We stappen even af van ons spleten-experiment en bekijken een nieuwe opstelling \cite[Feynman:100771,Klauber:2010tc]. In plaats van deeltjes rechtstreeks van een bron $B$ naar een detector $D$ te sturen, schieten we de elektronen richting een reflecterende plaat $r$. Klassiek gezien zou een deeltje het pad volgen waarbij de hoek van inval gelijk is aan de hoek van terugkaatsing. Hiervoor hebben we al gezien dat er veel meer paden mogelijk zijn. In \see[fig:reflecties] zijn, naast het klassieke pad, veertien van zulke paden getekend.
Voor elk pad $p_i$ kunnen we een actie $S_i$ uitrekenen met behulp van \see[for:actie]. In \see[fig:acties] is mooi te zien dat de actie bij pad $p_7$ minimaal is, dit is dan ook precies het klassiek pad. 

\placefigure[][fig:acties]
  {Acties bij elk pad $p_0$ tot en met $p_{14}$ zoals weergegeven in \see[fig:reflecties]. $p_7$ is minimaal en zodoende het klassieke pad van $B$ naar $D$.}
  {\externaltikzpicture[acties][scale=0.5]}

%Een som over alle paden kunnen we ons als volgt voorstellen.
%Elk pad $p_1,\dots,p_{14}$ heeft een bijbehorende fase $\phi_i$ gekoppeld aan de top van de golf zoals beredeneerd in \see[sec:fases]. De bijbehorende fasoren zijn getekend in \see[fig:padensom]. Wanneer we al deze fasoren optellen, komen we uit op de fasor van het klassieke pad.  

In de vorige sectie hebben we gezien dat de actie een maat is voor de fase op het moment dat een deeltje zijn pad heeft doorlopen. Door bij elk pad een fasor op te stellen en deze te sommeren vinden we de waarschijnlijkheidsamplitude om van $B$ naar $D$ te komen. Dit is weergegeven in \see[fig:padensom]. Hierbij vallen enkele dingen op:
\startitemize
\item Alle paden tellen even zwaar mee. De paden die dichter bij het klassieke pad liggen hebben \emph{geen} voorkeur ten opzichte van de paden die totaal niet klassiek zijn.
\item De relatie met het klassieke pad komt voort uit de destructieve interferentie \emph{van de fases} van de paden die ver van klassiek zijn en de constructieve interferentie bij de paden die bijna klassiek zijn.
\item De klok op elk pad loopt \emph{vooruit}. Dit is impliciet gedefinieerd in \see[for:actie], waarin $L$ geïntegreerd wordt over de tijd in positieve richting. We hebben hier ook al gebruik van gemaakt bij de afleiding van \see[for:faseverandering]. Deeltjes op een pad kunnen dus niet teruggaan in de tijd en vervolgens weer vooruit lopen.
\stopitemize

\placefigure[][fig:padensom]
  {}
  {\externaltikzpicture[padensom]}

\section[sec:meer spleten,sec:padintegraal]{Nog meer spleten}

We hebben genoeg informatie over padintegralen verzameld om een concrete afleiding te geven. Hiervoor keren we weer terug naar experimenten met spleten. In voorgaande secties vonden onze experimenten plaats in een twee dimensionale ruimte. Nu doen we een stapje terug en beschouwen een pad van een bron $B$ naar een detector $D$ in één dimensie. Ons deeltje bevind zich op tijdstip $t_B$ in $x_B$ en op $t_D$ in $x_D$. Om een pad te kunnen karakteriseren splitsen we de tijd op in $N$ \emph{snedes} van grootte $\Delta t$ zodat
  \placeformula[for:delta t]
  \startformula
  \Delta t := \frac{t_D - t_B}{N}.
  \stopformula
In \see[fig:tijdsnedes] is een mogelijk pad getekend om van bron naar detector.
We hebben het pad in de tijd uitgerekt, zodat we op ieder tijdstip $t_n$ kunnen aangeven waar het deeltje zich bevindt. Zo een positie is op te vatten als een \emph{spleet} in een scherm op snede $n$. Het is natuurlijk goed mogelijk dat het deeltje op een snede een \emph{andere} spleet had gekozen.\footnote{Lees: het deeltje heeft een ander pad afgelegd.} Hier komen we later op terug.
Merk op dat het deeltje, zoals we hebben beredeneerd in \see[sec:reflecties], niet terug kan in de tijd. Wel kan het heen en weer bewegen in de plaats en hoeft dus niet met constante snelheid van $B$ naar $D$ te reizen.

\placefigure[][fig:tijdsnedes]
  {}
  {\externaltikzpicture[tijdsnedes][yscale=0.5]}

Om de padintegraal af te leiden lopen we effectief de postulaten uit \see[sec:postulaten] in omgekeerde volgorde af. 
Uit het derde postulaat weten we dat de \emph{bijdrage} aan een pad proportioneel is met de actie. Wanneer we van tijdsnede $n$ naar snede $n+1$ gaan kunnen we de bijdrage $\psi_{n \to n+1}$ uitrekenen met
  \startformula
  \psi_{n \to n+1} = A \e^{i S / \hslash}.
  \stopformula
Wel moeten we de actie voor dit korte pad kunnen uitrekenen. Dit kan met een benadering van de Lagrangiaan,
  \define\LApprox{\math{L\left( \frac{x_{n+1} + x_n}{2}, \frac{x_{n+1} - x_n}{\Delta t}, \frac{t_{n+1} + t_n}{2} \right)}}
zodat
  \startformula
  S \approx \LApprox \Delta t.
  \stopformula
Dit is correct tot op eerste orde. De bijdrage voor een stukje pad tussen tijdstip $t_n$ en $t_{n+1}$ is dan bij benadering
  \startformula
  \psi_{n \to n+1} \approx A \exp\left[ \frac{i}{\hslash} \LApprox \Delta t \right].
  \stopformula
Maar we willen de bijdrage van het \emph{volledige} pad.
\todo{Anders} Aangezien we in \see[sec:postulaten] ook hebben gezien dat we de bijdrage als een fase kunnen opvatten, kunnen we de totale fase berekenen door de fases van de afzonderlijke tijdsnedes te sommeren:
\todo{Iets met die constante}
  \startformula\startsplit
  \psi_{0 \to N} \SC
  \approx A \exp\left[ \frac{i}{\hslash} \sum_{n=0}^{N-1} \LApprox \Delta t \right] \SR
  = A \prod_{n=0}^{N-1} \exp\left[ \frac{i}{\hslash} \LApprox \Delta t \right].
  \stopsplit\stopformula

Om de \emph{waarschijnlijkheidsamplitude} uit te rekenen bij de overgang van $B$ naar $D$ moeten we, volgens het tweede postulaat, de som nemen over \emph{alle mogelijke paden} van $x_B$ naar $x_D$. Tot nu toe hebben we alleen maar één pad uitgerekend. Daarbij zijn we er van uit gegaan dat ons pad op ieder tijdstip $t_n$ over vastgelegde punten $x_n$ loopt.\footnote{Ofwel: het deeltje gaat door specifieke spleten.} Deze plaats is natuurlijk willekeurig te kiezen uit \emph{alle mogelijke punten} $x$. Om elk mogelijk pad van $B$ naar $D$ te krijgen, moeten we dus integreren over alle mogelijke tussenpunten voor elke tijdsnede $t_n$. In \see[fig:tijdsnedes] kunnen we zien dat $t_0$ en $t_N$ hierop een uitzondering zijn, deze twee punten liggen immers vast. We krijgen dus $N-1$ keer een integraal over $x_n$ met $n$ van $1$ tot $N-1$. Dit leid tot
  \placeformula[for:integraal over alle punten]
  \startformula
  \psi \approx \int_{-\infty}^\infty \cdots \int_{-\infty}^\infty \int_{-\infty}^\infty \psi_{0 \to N} \d x_1 \d x_2 \;\cdots \d x_{N-1}.
  \stopformula

In \see[for:integraal over alle punten] hebben we nog te maken met een benadering. Door onze tijdsnedes steeds smaller te maken, krijgen we de correcte uitdrukking voor de waarschijnlijkheidsamplitude. We nemen dus de limiet van $N$ naar $\infty$ zodat $\Delta t$ naar $0$ gaat volgens de definitie van \see[for:delta t]. 
  \startformula\startsplit
  \psi \SC
  = \lim_{N \to \infty} \int_{-\infty}^\infty \cdots \int_{-\infty}^\infty \int_{-\infty}^\infty \psi_{0 \to N} \d x_1 \d x_2 \;\cdots \d x_{N-1} \SR
  =: \int_{x_B}^{x_D} \e^{i S / \hslash} \D x
  \stopsplit\stopformula
bla
  \startformula
  \psi = \lim_{\Delta t \to 0} A \prod_{n=0}^{N-1} \exp\left[ \frac{i}{\hslash} L\left( \frac{x_{n+1} + x_n}{2}, \frac{x_{n+1} - x_n}{\Delta t}, \frac{t_{n+1} + t_n}{2} \right) \Delta t \right].
  \stopformula

\stopcomponent

% vim: spell spl=nl cole=1

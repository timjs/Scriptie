\startcomponent voorwoord
\product        scriptie
\environment    thesislayout

\chapter[chp:voorwoord]{Voorwoord}

Aan het einde van je bachelor natuurkunde is het zo ver: je mag stage gaan lopen. Gedurende een half jaar wordt een van de afdelingen van de universiteit een deel van je leven. Je kunt kijken of je al die mooie, natuurkundige theorieën die je de afgelopen jaren heb mogen leren kunt toepassen \quote{in het wild}. Maar onderzoek doen is niet zomaar iets. Het is een enorme stad waarin je, als je even niet goed oplet, makkelijk verdwaalt. Meerdere malen manoeuvreer je in een grote laan, wat uiteindelijk een smal steegje blijkt te zijn. Soms loop je zelfs, geheel uit het niets, tegen een harde muur aan. Maar soms ontdek je ook mooie parken, waar je even kunt genieten van het uitzicht om daarna weer door te gaan, op jacht naar\dots\ Ja, wat eigenlijk precies?

Deze scriptie laat verschillende delen van zo'n tocht zien. In \see[chp:paden] bekijken we een van de mooiste theorieën van de natuurkunde: de veel-paden formulering van de kwantummechanica. We lopen stap voor stap langs de zuilen van deze kathedraal, totdat we de Feynman padintegraal zelf kunnen afleiden. In \see[chp:reflecties] laten we dit wiskundige bouwwerk achter ons, en bestuderen we een methode om inzicht te krijgen in de padintegraal. Hoe kan een som over alle paden iets nuttigs opleveren? Wat houdt destructieve en constructieve interferentie precies in bij de padintegraal? Na te hebben genoten van dit beeld moeten we door. In \see[chp:normalisatie] gaan we op jacht naar een manier om onze inzichten van \see[chp:reflecties] vast te leggen. Dit blijkt geen eenvoudige opgave. Sommige straten lopen dood, andere blijken te nauw, weer andere blijken in de verkeerde wijk te liggen.

Een stad als deze kun je bijna niet in je eentje verkennen. Daarom wil ik dr.~Wim Beenakker bedanken die nu eens op een straathoek stond te wachten met een lantaarn om me de weg te wijzen, en dan weer met zijn kennis mij een tocht naar de bibliotheek bespaarde. Daarbij heeft hij, samen met dr.ir.~Gilles de~Wijs, de tijd en ruimte gevonden om deze scriptie te lezen, van commentaar te voorzien en te beoordelen. Daarnaast zijn er natuurlijk ook vele vrienden, bij wie je kunt aankloppen als je weer eens tegen een muur bent aangelopen of met wie je samen de stad kunt verkennen.

\startlines
Tim Steenvoorden
augustus 2012
\stoplines

\nocite[Beenakker:2009te]

\stopcomponent

% vim: ft=context spell spl=nl cole=1
